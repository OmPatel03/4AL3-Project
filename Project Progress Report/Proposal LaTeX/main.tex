\documentclass[11pt]{article}

% Change "review" to "final" to generate the final (sometimes called camera-ready) version.
% Change to "preprint" to generate a non-anonymous version with page numbers.
\usepackage[]{acl}
\usepackage{times}
\usepackage{latexsym}

% For proper rendering and hyphenation of words containing Latin characters (including in bib files)
\usepackage[T1]{fontenc}
% For Vietnamese characters
% \usepackage[T5]{fontenc}
% See https://www.latex-project.org/help/documentation/encguide.pdf for other character sets

% This assumes your files are encoded as UTF8
\usepackage[utf8]{inputenc}
\usepackage{microtype}
\usepackage{inconsolata}
\usepackage{graphicx}

% If the title and author information does not fit in the area allocated, uncomment the following
%
%\setlength\titlebox{<dim>}
%
% and set <dim> to something 5cm or larger.

\title{Group X Progress Report:\\My Group's Project Name}


\author{First Author, Second Author, Third Author \\
  \texttt{\{macid1,macid2,macid3\}@mcmaster.ca} }

%\author{
%  \textbf{First Author\textsuperscript{1}},
%  \textbf{Second Author\textsuperscript{1,2}},
%  \textbf{Third T. Author\textsuperscript{1}},
%  \textbf{Fourth Author\textsuperscript{1}},
%\\
%  \textbf{Fifth Author\textsuperscript{1,2}},
%  \textbf{Sixth Author\textsuperscript{1}},
%  \textbf{Seventh Author\textsuperscript{1}},
%  \textbf{Eighth Author \textsuperscript{1,2,3,4}},
%\\
%  \textbf{Ninth Author\textsuperscript{1}},
%  \textbf{Tenth Author\textsuperscript{1}},
%  \textbf{Eleventh E. Author\textsuperscript{1,2,3,4,5}},
%  \textbf{Twelfth Author\textsuperscript{1}},
%\\
%  \textbf{Thirteenth Author\textsuperscript{3}},
%  \textbf{Fourteenth F. Author\textsuperscript{2,4}},
%  \textbf{Fifteenth Author\textsuperscript{1}},
%  \textbf{Sixteenth Author\textsuperscript{1}},
%\\
%  \textbf{Seventeenth S. Author\textsuperscript{4,5}},
%  \textbf{Eighteenth Author\textsuperscript{3,4}},
%  \textbf{Nineteenth N. Author\textsuperscript{2,5}},
%  \textbf{Twentieth Author\textsuperscript{1}}
%\\
%\\
%  \textsuperscript{1}Affiliation 1,
%  \textsuperscript{2}Affiliation 2,
%  \textsuperscript{3}Affiliation 3,
%  \textsuperscript{4}Affiliation 4,
%  \textsuperscript{5}Affiliation 5
%\\
%  \small{
%    \textbf{Correspondence:} \href{mailto:email@domain}{email@domain}
%  }
%}

\begin{document}
\maketitle
% \begin{abstract}
% \end{abstract}

\section{Introduction}

Online dating systems usually optimize for clicks or stated preferences, not mutual compatibility. We study the problem of predicting an individual’s ideal partner profile from structured demographic, preference, and ratings data. Given a person A, our model first predicts the set of people A would likely say "yes" to, then composes a representative partner profile B* from the top-k candidates. Next, it predicts who B* would likely say "yes" to, producing a second composite C*. The difference between A and C* captures a preference gap that reveals why A’s most desired partners may prefer someone different. This framing supports both recommendation and self-insight, and it generalizes to other matching settings such as mentorship and team formation.

We focus on the Columbia Speed Dating dataset, a tabular dataset with 8,378 rows and 123 columns, and formulate the task as classification for the yes or no decision with ranking objectives layered on top for recommendation quality. Our goals are: (1) build a leakage-free, explainable pipeline that predicts A’s decision on B, (2) construct B* and C* to surface preference gaps, and (3) evaluate with Accuracy, F1, ROC-AUC, and Precision@k.

\section{Related Work}

Our work is informed by several research areas. The first is the original analysis of our dataset, the Columbia Speed Dating Experiment, where Fisman and Iyengar found that attractiveness, fun, and shared interests were highly predictive of matching decisions. Second, our A→B→C loop concept is inspired by two-sided matching theory, most famously represented by the Gale-Shapley algorithm, which underscores that stable matches must account for the preferences of both sides. Third, our approach is related to the broader field of recommender systems, which often use collaborative or content-based filtering to predict preferences, though typically not in a two-hop "preference gap" framework. Fourth, we draw from psychological research on partner preferences, such as the meta-analysis by Finkel, Eastwick, and colleagues, which reviewed the predictive validity of stated ideal partner preferences. Finally, given our goal of providing insights, our work connects to explainable AI (XAI) methods, such as SHAP, which we plan to use to identify which features most influence compatibility predictions.
\section{Dataset}

We are using the Columbia Speed Dating dataset, as described in our proposal. It consists of \textbf{8,378 observations} (individual dates) and \textbf{123 columns}.
\subsection{Preprocessing and Cleaning}

The raw dataset required significant preprocessing. Our pipeline performs the following steps:
\begin{itemize}
    \item \textbf{String Decoding:} Many text columns were encoded as Python byte literals (e.g., `b'female'`). We decoded these into standard UTF-8 strings.
    \item \textbf{Normalization:} All string values were converted to lowercase and stripped of leading/trailing whitespace to ensure consistency (e.g., "Law" and "law" are treated as identical).
    \item \textbf{Missing Values:} We unified various missing value markers (e.g., "na", "n/a", "", "nan") into a single `pd.NA` representation.
    \item \textbf{Numeric Coercion:} Columns that appeared to be numeric but were stored as objects (e.g., "1.0", "5") were automatically coerced into floating-point types, while preserving categorical ranges (e.g., "[0-1]").
\end{itemize}

This stable ID generation is crucial for our participant-based train-test split.

\section{Features}

Describe any features you used for your model, or how your data was input to your model. Are you doing feature engineering or feature selection? Are you learning embeddings? Is it all part of one neural network? Refer to item 2. This may range from 0.25 pages to 0.5 pages.

\section{Implementation}

Describe your model and implementation here. Refer to item 4. This may take around a page.

\section{Results and Evaluation}

How are you evaluating your model? What results do you have so far? What are your baselines? Refer to item 5. This may take around 0.5 pages.

\section{Feedback and Plans}

Write about your plans for the remainder of the project. This should include a discussion of the feedback you received from your TA, and how you plan to improve your approach. Reflect on your implementation and areas for improvement. Refer to item 6. This may be around 0.5 pages.

\section{Template Notes}

You can remove this section or comment it out, as it only contains instructions for how to use this template. You may use subsections in your document as you find appropriate.

\subsection{Tables and figures}

See Table~\ref{citation-guide} for an example of a table and its caption.
See Figure~\ref{fig:experiments} for an example of a figure and its caption.


\begin{figure}[t]
  \includegraphics[width=\columnwidth]{example-image-golden}
  \caption{A figure with a caption that runs for more than one line.
    Example image is usually available through the \texttt{mwe} package
    without even mentioning it in the preamble.}
  \label{fig:experiments}
\end{figure}

\begin{figure*}[t]
  \includegraphics[width=0.48\linewidth]{example-image-a} \hfill
  \includegraphics[width=0.48\linewidth]{example-image-b}
  \caption {A minimal working example to demonstrate how to place
    two images side-by-side.}
\end{figure*}


\subsection{Citations}

\begin{table*}
  \centering
  \begin{tabular}{lll}
    \hline
    \textbf{Output}           & \textbf{natbib command} & \textbf{ACL only command} \\
    \hline
    \citep{Gusfield:97}       & \verb|\citep|           &                           \\
    \citealp{Gusfield:97}     & \verb|\citealp|         &                           \\
    \citet{Gusfield:97}       & \verb|\citet|           &                           \\
    \citeyearpar{Gusfield:97} & \verb|\citeyearpar|     &                           \\
    \citeposs{Gusfield:97}    &                         & \verb|\citeposs|          \\
    \hline
  \end{tabular}
  \caption{\label{citation-guide}
    Citation commands supported by the style file.
  }
\end{table*}

Table~\ref{citation-guide} shows the syntax supported by the style files.
We encourage you to use the natbib styles.
You can use the command \verb|\citet| (cite in text) to get ``author (year)'' citations, like this citation to a paper by \citet{Gusfield:97}.
You can use the command \verb|\citep| (cite in parentheses) to get ``(author, year)'' citations \citep{Gusfield:97}.
You can use the command \verb|\citealp| (alternative cite without parentheses) to get ``author, year'' citations, which is useful for using citations within parentheses (e.g. \citealp{Gusfield:97}).

\subsection{References}

\nocite{Ando2005,andrew2007scalable,rasooli-tetrault-2015}

Many websites where you can find academic papers also allow you to export a bib file for citation or bib formatted entry. Copy this into the \texttt{custom.bib} and you will be able to cite the paper in the \LaTeX{}. You can remove the example entries.

\subsection{Equations}

An example equation is shown below:
\begin{equation}
  \label{eq:example}
  A = \pi r^2
\end{equation}

Labels for equation numbers, sections, subsections, figures and tables
are all defined with the \verb|\label{label}| command and cross references
to them are made with the \verb|\ref{label}| command.
This an example cross-reference to Equation~\ref{eq:example}. You can also write equations inline, like this: $A=\pi r^2$.


% \section*{Limitations}

\section*{Team Contributions}

Write in this section a few sentences describing the contributions of each team member. What did each member work on? Refer to item 7.

% Bibliography entries for the entire Anthology, followed by custom entries
%\bibliography{custom,anthology-overleaf-1,anthology-overleaf-2}

% Custom bibliography entries only
\bibliography{custom}

% \appendix

% \section{Example Appendix}
% \label{sec:appendix}

% This is an appendix.

\end{document}
